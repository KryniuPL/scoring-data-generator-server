\documentclass[12pt, twoside]{article}
\usepackage[a4paper, top=2.0cm, bottom=2.0cm, left=2.5cm, right=2.5cm]{geometry}
\usepackage[MeX]{polski}
\usepackage[T1]{fontenc}
\usepackage[utf8]{inputenc}
\usepackage{color}
\usepackage{indentfirst}
\usepackage{graphicx}
\usepackage{mathptmx}
\usepackage{amsmath}
\usepackage{tikz}
\usepackage{hyperref}
\usepackage[final]{pdfpages}
\usepackage{afterpage}
\usepackage{titlesec}
\usepackage{microtype}
\usepackage{enumitem}
\usepackage{caption}
\usepackage{subcaption}
\usepackage{listings}
\usepackage{chngcntr}

\lstset{
    basicstyle=\ttfamily\footnotesize,
    columns=fullflexible,
    showstringspaces=false,
    numbers=left,                   % where to put the line-numbers
    numberstyle=\tiny\color{gray},  % the style that is used for the line-numbers
    stepnumber=1,
    numbersep=5pt,                  % how far the line-numbers are from the code
    backgroundcolor=\color{white},      % choose the background color. You must add \usepackage{color}
    showspaces=false,               % show spaces adding particular underscores
    showstringspaces=false,         % underline spaces within strings
    showtabs=false,                 % show tabs within strings adding particular underscores
    frame=single,                   % adds a frame around the code
    rulecolor=\color{black},        % if not set, the frame-color may be changed on line-breaks within not-black text (e.g. commens (green here))
    tabsize=2,                      % sets default tabsize to 2 spaces
    captionpos=b,                   % sets the caption-position to bottom
    breaklines=true,                % sets automatic line breaking
    breakatwhitespace=false,        % sets if automatic breaks should only happen at whitespace
    title=\lstname,                   % show the filename of files included with \lstinputlisting;
% also try caption instead of title
    commentstyle=\color{gray}\upshape
}

\ProvidesPackage{java}

\title{Projekt generatora danych skoringowych}
\author{Krzysztof Dragan}
\date{Listopad 2022}
\pdfinfo{
    /Author (Krzysztof Dragan)
    /Title  (Generator danych skoringowych z użyciem Apache Kafka)
    /CreationDate (\today)
    /Keywords (Praca;dyplomowa;magisterska;modyfikacja;stfc)
}

\begin{document}

    \maketitle

    \newpage
    \section{Opis zagadnienia}
    Głównym zagadnieniem podjętym w pracy jest opracowanie generatora danych skoringowych z użyciem oprogramowania Apache Kafka. Na domenę biznesową wybrano system oceny kredytowej klientów indywidualnych wzorujący się na realnym systemie Biura Informacji Kredytowej.
    Wymgania i cele pracy można podzielić na dwie kategorie: biznesowe i techniczne. Wymagania techniczne zawierają potrzeby zaimplementowania systemu:
    \begin{itemize}
        \item skalowalnego
        \item dostępnego z poziomu przeglądarki internetowej
        \item pozwalającego wygnerować duże ilości danych(minimum 1GB)
        \item wydajnego, pozwalającego na wygenerowanie danych w krótkim czasie
        \item parametryzowanego, dostarczającego wielu opcji na określenie zakresu atrybutów zbioru
    \end{itemize}
    Co do wymagań biznesowych, system powinien:
    \begin{itemize}
        \item jak najbardziej sensownie odzwierciedlać powiązania danych
        \item w łatwy sposób udostępniać wyeksportowanie danych dla użytkownika końcowego systemu
    \end{itemize}

\end{document}
